\documentclass[doc,natbib,12pt]{article}

\usepackage[american]{babel}
\usepackage[utf8x]{inputenc}
\usepackage{amsmath}
\usepackage{graphicx}
\usepackage[colorinlistoftodos]{todonotes}
\usepackage{varioref}
\usepackage[hidelinks]{hyperref}
\usepackage[T1]{fontenc}

\title{CISC 611-90-O-2019/Late Spring - Network Operating Systems \\ Homework - 3}

\author{Youwei Lu}
\date{}

%\abstract{Referring to Operating Systems architecture, the topic of this assignment is on the ``Memory Management'' which is part of the computer's physical memory or Random-Access Memory (RAM) organization.}

\begin{document}
	\maketitle
	
	
	\subsection*{Exercise 19.15: show how to construct a send-constrained channel from a receive-constrained channel, and vice versa. Hint: use a trusted node connected to the given channel.}
	
	The following nomenclatures are used in the solution:
	\begin{itemize}
		\item $c$: Any channel connecting the parties that to be able to communicate
		\item $h$: Hash
		\item $m$: Message
		\item $N$: Trusted node
		\item $t$: Time by $N$'s clock
		\item $rc$: Receive-constrained channel
		\item $sc$: Send-constrained channel
	\end{itemize}

	With the nomenclatures used, the problem can be rephrased as: given a $rc$, show how to construct a $s(rc)$.
	
	Use a $N$. when it receives a $m$, it uses the received-constrained channel $rc$ to return to the sender a signed hash $\text{sig}\{h(m,t)\}$. Then construct $s(rc)$ from $c$ and $N$. The rules for sending and receiving on $s(rc)$ are as follow:
	
	To send $m$ on $s(rc)$:
	
	\hspace{0.628in} send $m$ to $N$
	
	\hspace{0.628in} receive $< t$, $\text{sig}\{h(m,t)\}$ > from $N$ over $rc$
	
	\hspace{0.628in} send $< m, t$, $\text{sig}\{h(m,t)\}$ > on $c$.
	
	To receive $m$ on $s(rc)$:
	
	\hspace{0.628in} receive $< m, t$, $h > $ on $c$
	
	\hspace{0.628in} verify $h = \text{sig}\{h(m,t)\}$
	
	\hspace{0.628in} verify currency of $t$ and freshness of $h$
	
	\hspace{0.628in} Discard $m$ if verification fails, else receive $m$.
	
	It is assumed that the receivers' clocks are synchronized to $N$'s clock. The timestamp (and hence state of the sender) is deemed current if it is within a given bound of the time on the receiver's clock. To prevent replay attacks, the receiver need remember the hashes for only a limited time: older messages will have a non-current timestamp.
	
	In a similar fashion, a receive-constrained channel $r(sc)$ can be implemented from a send-constrained channel $sc$. All messages are sent (over any channel) to a trusted node, which stores them. Receivers must use a particular send-constrained channel to reach that node, which responds with the next message for them.
	
	
\end{document}

%
% Please see the package documentation for more information
% on the APA6 document class:
%
% http://www.ctan.org/pkg/apa6
%